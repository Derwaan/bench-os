\documentclass[journal, a4paper]{../IEEEtran}

% Packages
\usepackage{graphicx}
\usepackage[utf8]{inputenc}
\usepackage[T1]{fontenc}
\usepackage{hyperref}
\usepackage{amsmath}



\begin{document}

% Title - Author - Date
\title{RIOT Summit 2018 -- Report}
\author{Julien Gomez \and Trong-Vu Tran}
\date{September 13-14 2018}
\markboth{Université catholique de Louvain}{}
\maketitle

% ABSTRACT
\begin{abstract}
	During the RIOT Summit 2018, we talked to some developers that were interested in our thesis. 
    A good approach for our work would be to contact the IoT community and ask them in what they are interested in the benchmarking of real-time OS. 
    During the first day, we discover other RTOS that could be benchmarked: \textit{Apache Mynewt}, \textit{Zephyr}, \textit{MbedOS}, \textit{Contiki-NG}. 
    During the second day, we discover service like \textit{The Things Network} and \textit{IoT Lab}.
\end{abstract}

%% INTRODUCTION
\section{Introduction}
\IEEEPARstart{F}{or} our master thesis "Benchmarking of Real-Time Operating Systems on Internet of Things", we went to the RIOT Summit 2018 at Amsterdam.
    
This was the third summit of the RIOT community. 
Every year the members of this community regroup and talk about their projects and the future work for RIOT-OS\footnote{\url{http://riot-os.org}}. \\

The summit was divided into two days. 
During the first day, 12 speakers presented their work. On the second day, tutorials were given and breakout groups ended the summit. 
More information can be found on the summit \href{http://summit.riot-os.org/2018/}{website}. \\

This report will present what we learned during these two days and what will be useful for our master thesis.

\section{First day}
During the first day, 12 talks were given by IoT developers and enthusiasts.
For each of the keynotes, we describe what we learned and what will be useful for our thesis.

Notice that we do not mention all keynotes as some of them are irrelevant for our master thesis.

\subsection{From R\&D to production -- Jaime Jiménez}
In the IoT world, many standards exist. In the connectiviy layer, we can found \textsc{6LoWPAN}, \textsc{LoRaWAN}, \textsc{ROLL}, \dots \\
In the applications layer, \textsc{CoAP}, \textsc{CBOR}, \dots \\ 
In the security layer, \textsc{COSE}, \textsc{DICE}, \dots \\

The speaker shows us that the IoT world is shared among four major "players":
\begin{itemize}
    \item System integrators,
    \item Cloud providers,
    \item Telecom infrastructures and
    \item Operators.
\end{itemize}

Between these major players, there is still room for innovative startups. Jaime Jiménez show us some example like \textit{Smart Rockbolt} or \textit{IOTEROP}. \\

In conclusion, we learn that the IoT world is complex with a lot a standards. 
Many of these standards are open-source.
This help startups to immerge from a highly competitive world among big players like Google, IBM or AT\&T. \\

Slides are available \href{http://summit.riot-os.org/2018/wp-content/uploads/sites/10/2018/09/0_3-Jaime-Jimenez-Keynote.pdf}{here}.

\subsection{NimBLE - Portable bluetooth stack from Apache Mynewt -- Szymon Janc}
From this keynote, we discore a new RTOS: \textit{Apache Mynewt}.

% NimBLE can be interesting to benchmark
% Bluetooth has goes trough a lot of improvements
% BLE = Bluetooth Low Energy - build from scratch

Slides are available \href{http://summit.riot-os.org/2018/wp-content/uploads/sites/10/2018/09/1_1-Szymon-Janc-NimBLE.pdf}{here}.

\subsection{Google Protocol Buffers for embedded IoT -- Morgan Kita}

% JSON is too hard for IoT (parsing)
% Pro and Con of GPB
% GPB implemented by Nano-PB
% RTOS using JSON or XML can be improved

Slides are available \href{http://summit.riot-os.org/2018/wp-content/uploads/sites/10/2018/09/1_2-Kita-Morgan-Protobuf.pdf}{here}.

\subsection{Resource discovery, Object Security and other news from CoAP -- Christian Amsüss}
This talk gives an introduction to the CoAP protocol and its recent and ongoing developments mainly in term of security and network topology.\\

The first part of the talk was about the CoRE Resource Directory, a centralized directory where servers (i.e. devices) can register their resources.
Clients can then request a particular resource with a single request to the RD (i.e. Resource Directory).
He then explained in more details the aspects and advantages of this topology.\\

The second part of the talk was about OSCORE (Object Security for CoRE).


% RD = search engine in IoT
% Proxy used to retransit packets
% OSCORE = security in CoAP
% OSCORE use POST req. with encrypted payload
% CoAP too high level for us

Slides are available \href{http://summit.riot-os.org/2018/wp-content/uploads/sites/10/2018/09/1_3-Christian-Asmuess-CoAP-RD.pdf}{here}.

\subsection{Build a robot with RIOT -- Gilles Doffe}
For the Robotic Cup 2018, \textit{Savoir-faire Linux} built a robot using RIOT-OS.
The main part we will focus on for this keynote is the architecture and the scheduling they use for their robot.
They implemented three threads: one for the motion control, one for the planner and one for the analog sensors.
 
The \textit{Savoir-faire linux} team struggled with the scheduler of RIOT-OS. 
The RTOS provides a tickless, preemptive and priority based scheduler.

After speaking with Gilles Doffe, we understand more how the scheduler of RIOT-OS works.
For our thesis, we will take into account the type of scheduler used by the benchmarked RTOS.


% Scheduler of RIOT is tickless (no clock to check for priority)
% Issues with real app: constraint memory, scheduling, clock period, tasks
%   taking too much time
% Need to take into account the type of the OS scheduler

Slides are available \href{http://summit.riot-os.org/2018/wp-content/uploads/sites/10/2018/09/2_1-Gille-Doffe-RIOT-Robot.pdf}{here}.

\subsection{New Crypto-fundamentals in RIOT -- Pieter Kietzmann}

The speaker talked about the lack of secured hardware in Commercial Off-the-Shelf IoT devices such as Intel SGX or ARM TrustZone.
Those devices require efficient security softwares implementations that can be contained within their constraints.\\

He presented what is called Physical Unclonable Functions or PUF, 
    functions that are based of physical characteristics of a particular device (such as components delays, magnetism or some patterns in the uninitialized memory).

With those kind of functions, a single device own an unique fingerprint which is different from another device (even from the same batch) due to variations in the manufacturing process. \\

He then explained the problematic of noise in the process of generating and authenticating a device using PUFs.
After that, he presented his implementation of a SRAM based PUF using RIOT for generating PRNG seeds and keys. \\

What we can retain from this presentation is that implementing entropy using a constrained device (for key or seed generation for example) is not easy. \\
Benchmarking and analyzing the current implementations PRNG's in RTOS's could be very interesting (distribution, bias, uniqueness). \\

Slides are available \href{http://summit.riot-os.org/2018/wp-content/uploads/sites/10/2018/09/3_2-Peter-Kietzmann-Crypto-Fundamentals.pdf}{here}.

\section{Second day}

% Tutorials on IoT Lab + The Things Network
% Philip repo
% Hardware in the Loop testing

\section{Additional information}

% STM32F4 - ST series, Cortex M4
% Confirmation for the logical analyser approach


%% CONCLUSION
\section{Conclusion}
\end{document}
