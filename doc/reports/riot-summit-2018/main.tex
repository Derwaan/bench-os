\documentclass[journal, a4paper]{../IEEEtran}

% Packages
\usepackage{graphicx}
\usepackage[utf8]{inputenc}
\usepackage[T1]{fontenc}
\usepackage{url}
\usepackage{amsmath}



\begin{document}

% Title - Author - Date
\title{RIOT Summit 2018 -- Report}
\author{Julien Gomez \and Trong-Vu Tran}
\date{September 13-14 2018}
\markboth{Université catholique de Louvain}{}
\maketitle

% ABSTRACT
\begin{abstract}
	During the RIOT Summit 2018, we talk to some developers that were
	interested in our thesis. A good approach for our work would be to
	contact the IoT community and ask them in what they are interested in
    the benchmarking of real-time OS. During the first day, we discore other 
    RTOS that could be benchmarked: \textit{Apache Mynewt}, \textit{Zephyr}, 
    \textit{MbedOS}, \textit{Contiki-NG}. During the second day, we discover 
    service like \textit{The Things Network} and \textit{IoT Lab}.
\end{abstract}

%% INTRODUCTION
\section{Introduction}
\IEEEPARstart{F}{or} our master thesis "Benchmarking of Real-Time Operating 
Systems on Internet of Things", we went to the RIOT Summit 2018 at Amsterdam.
    
This was the third summit of the RIOT community. Every year the members of
thi community regroup and talk about their projects and the future work for
RIOT-OS\footnote{\url{http://riot-os.org}}.

The summit was divided into two days. During the first day, 12 speakers
presented their work. On the second day, tutorials were given and breakout
groups ended the summit. More information can be found on the summit 
website\footnote{\url{http://summit.riot-os.org/2018/}}.

This report will present what we learn during these two days and what will
be useful for our master thesis.

\section{First day}
During the first day, 12 talks were given by IoT developers and enthusiasts.
For each of the keynote, we describe what we learn and what will be useful for
our thesis.

Notice that we do not mention all keynotes as some of them are irrelevant for
our master thesis.

\subsection{From R\&D to production -- Jaime Jiménez}
The speaker shows us that the IoT world is shared among four major "players":
\begin{itemize}
    \item System integrators,
    \item Cloud providers,
    \item Telecom infrastructures and
    \item Operators.
\end{itemize}

% Room left for startups
% Lot of standars: Connectivity, applications and security
% Open-source is difficult with patents

\subsection{NimBLE - Portable bluetooth stack from Apache Mynewt -- Szymon Janc}
From this keynote, we discore a new RTOS: \textit{Apache Mynewt}.

% NimBLE can be interesting to benchmark
% Bluetooth has goes trough a lot of improvements
% BLE = Bluetooth Low Energy - build from scratch

\subsection{Google Protocol Buffers form embedded IoT -- Morgan Kita}

% JSON is too hard for IoT (parsing)
% Pro and Con of GPB
% GPB implemented by Nano-PB
% RTOS using JSON or XML can be improved

\subsection{Resource discovery, Object Security and other news from CoAP -- 
Christian Amsüss}

% RD = search engine in IoT
% Proxy used to retransit packets
% OSCORE = security in CoAP
% OSCORE use POST req. with encrypted payload
% CoAP too high level for us

\subsection{Build a robot with RIOT -- Gilles Doffe}
 
% Scheduler of RIOT is tickless (no clock to check for priority)
% Issues with real app: constraint memory, scheduling, clock period, tasks
%   taking too much time
% Need to take into account the type of the OS scheduler

\subsection{New Crypto-fundamentals in RIOT -- Pieter Kietzmann}

% Entropy in RIOT not easy
% Physical Unclonable Function to generate seed or key
% Entropy can be a criteria

\section{Second day}

% Tutorials on IoT Lab + The Things Network
% Philip repo
% Hardware in the Loop testing

\section{Additional information}

% STM32F4 - ST series, Cortex M4
% Confirmation for the logical analyser approach


%% CONCLUSION
\section{Conclusion}
\end{document}
