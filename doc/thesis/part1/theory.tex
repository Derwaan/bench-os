\part{Theory of RTOS}

\chapter{RTOS characteristics}

\paragraph{}
In order to understand how an RTOS works compared to a general purpose operating system, it is essential to define some reccurent characteristics found in RTOS's.
\\
This chapter is of course non exhaustive and one can debate about the importance of one relative to one other.
We decided to chose those characteristics because they are commonly used in the litterature and we think that they will provide a good glance at what we can expect from an RTOS.


\section{Kernel Architecture}
\section{Scheduling}
\section{Programming Model}
\section{Memory Management}
\section{Hardware Support}
\section{Energy Management}
\section{File System}
\section{Modularity}
\section{Application Level}
\section{Kernel Level}


\chapter{RTOS Outline}

\section{RIOT OS}
\section{Contiki}